\documentclass[runningheads]{llncs}
\usepackage{graphicx}
\begin{document}
\title{Multisignal digital biosensors- digital Biosensors integrated with enzyme logic systems}
\author{Maren Krafft}
\authorrunning{F. Author et al.}
\institute{Universität Passau, Passau, Germany}
\maketitle        

      
\begin{abstract}
	
% describes central points of your report

\begin{itemize}
	\item biochemical logic systems
	\item 
\end{itemize}



\keywords{Biomolecular computing \and Biosensors  \and enzyme logic gates \and biosensing \and biocomputing systems \and enzyme logic circuits \and biomedical applications}
\end{abstract}

\section{Introduction}
%1.Context: what problem is being solved, why relevant or interesting, benefit, possible applications
%2.innovation: described technique completely new or does it improve earlier approcahes? what is improved 
%3.thesolution: how does the presented technique work, core idea of the solution

\subsection{Context}

	In the medical field Biosensors, analytic devices to convert a biological response into a electric signal, are a essential tool for monitoring and detection of a wide range of medical conditions from Diabetes to .....
	While common biosensing devices are limited to a single input, the novelty of Biosensors based on enzyme-based logic systems can process muliple biochemical signals. This article concentrates on the concept of the mulisignal processing Biosensors and the resulting challenges.\\
	
	Profound impact\\
	
	Through processing automatically several biochemical inputs(physiological information), it can provide a rapid and reliable assessment of overall physiological conditions. This can help a optimal timley therapeutic intervention. They will realize sense/delivery feedback loops by coupling signal processing with chemical actuators to revolutionize patient monitoring and drug delivery. 
	
	In the Biosensors processing multiple biochemical signal, the core idea is to add a biocomputing layer that produces a final output in form of a yes/no response. Kapitel 1.2\\
	
	Chances:
	
	
	
	
	In contrast to recent biosensors, those with a 11111111111 logic promise a higher fidelity, a greater range of processable inputs, more complex applications such as sense-act-treat loops and rapid assessment of the respective substances.(mehr ausformulieren)
	



\section{concept}

To understand the concept of biocomputing biosensors it is fundamental to know the basics about biosensors and enzyme-based logic systems. 
%\subsection{Biosensors: layers}
	A Biosensor is a device that transforms biochemical information, into an analytically useful signal. It contains usually the two basic components, the receptor and a tranducer. 
	
	convert a biocolical response into an electrical signals. the receptor can recognice biological moleculas include enzymes and the transducer senses physical changes and converts it into an electrical signal
		
		

\subsection{Biocomputing: concept}
	
	\subsubsection{allgemein}
	In Biocomputing the reaction of different biomolecular tools, including enzyme, resulting in a desired end product is used to mimick Boolean logic gates such as AND and OR. To digitalize chemical processes two levels of concentrations of chemical reaction materials (enzymes) are considers as input signals. 0 is usually considered as the absence of a enzyme, but it can be altered. 1 equals a  significantly difference to the absence or the as 0 defined concentration. 
	Output signal ein vorher bestimmter stoff, falls dieser vorhanden regel 1 falls nicth regel 0.  
	(Katz) 
	In \label{img:and} glucose oxidase and catalase operate as the logic gate machinery. The two input signals H2O2 and glucose. When both substrates present the inputs reacted and produces gluconic acid and this optical absorbance change was defined as the ouput signal of the enzyme logic gate, mimicking the Boolean Operation AND. 

	\begin{figure} \centering \includegraphics[scale= 0.3]{AND.png} \caption{Network diagramm} \label{img:and} \end{figure}
	By assembling these single logic gates, mimicking Boolean operations,it is possible to create small logic networks( e.g. hald-adder/ half subtractor)
	
	
	\begin{figure} \centering \includegraphics[scale= 0.2]{biocomputing_sensor.png} \caption{Network diagramm} \label{img:grafik-test} \end{figure}
	
	analysis of the chemical output signals generated by biomolecular logic systems is often limited.
	to digitalize chemical process two levels of concentrations of chemical reaction species are considers as input signals. 0 = usually absense of enzyme. 1 significantly different to absence. the chemical inputs are applied in different binary combinations (Katz) 
		\begin{itemize}
			\item biomolecular logic systems: definiton
			\item subarea of chemical computing 
			\item single logic gates to small logic networks (for example half-adder/ half subtractor)
			\item biomolecular systems for processing chemical information
			\item different biomolecular tools (including proteins/enzymes) assemble biocomputing systems processing biochemical information \\	
		\end{itemize}
	\subsubsection{what would bring it to biosensors}	
	\begin{itemize}	
		\item Enzyme logic system: multiassemblies to perform simple arithmetic functions
		\item idea: applicatoin of biomolecular logic system for analytical purpose new class of biosensors that accept many input signals and produce binary outputs in form yes/no 
		\item example analyse protein libraties associated with muliple sclerosis(58)
	\end{itemize}

	\subsubsection{how enzme-based logic gates work}	
Enzyme logic gates
\begin{itemize}
	\item enzymatic reactions
	\item coupling of logic gates with electronic transducers and signal responsive materials
	\item transducers:
	\item signal responive material
	
\end{itemize}
example theoretical with graphics

\begin{itemize}
	\item glucose oxidase and catalyse operating as logic gates: 
	\item input : H2O2 and glucose
	\item gluconic acid = biocalatyltic oxidation of glucose 
	\item only when both present opical output signal. = AND
	\item define logic values: small changes = 0 and large absorbance changes as 1 => AND
	\item similar possible with XOR,AND, OR, NOR, INHIBIT
	\item with logic gates with modular structur that enables therir assembly in networks NAND/ NOR possible
	\item logic gates and their networks = biomolecular information processing systems 
	\item => biosensoric systems with logically processed signals represented by various biomarkers(characteristic of different abnormal physiological conditions)
\end{itemize}

example ph
\begin{itemize}
	\item pH changes in solution as logic respones to input signals
	\item AND invertase + glucose oxidase (from 5.8 to 3.5)
	\item OR ersterase and glucose oxidase in glucose and ethyl butyrate - when one of both present ->acidification  
	\item neutral ph = 3,5
\end{itemize}
Conclusion: 
\begin{itemize}
	\item don't solve real computing problem  nor operate as useful biosensors 
	\item represent first step toward the development of digital biosensors 
	\item funfact optimization of enzymatic reaction, up to 10 logic gates concatenated with low noise in the system 
\end{itemize}

	\subsubsection{for biochemical analytic applications}
		\begin{itemize}
			\item design of biosensoric systems with logically processes signals represented by varous biomarkers characteristic for different abnormal physiological conditions
		\end{itemize}


\subsection{Biosensors  logic systems}
		Biosensors logically processing multiple biochemical signals\\
		-such procassed information produces a final output yes/no \\
		- boolean logic networks composed of biomolecular systems\\	
		\begin{itemize}
			\item multiple target analytes(inputs) for enzyme gates
			\item high-fidelity compared
			\item closed loop/feedback loops possible (sense/act/treat)
			\item rapid and reliable assessment of overall physiological condition
			\item could initiate optimal timely therapeutic intervention
			\item biosensors + enzyme logic gates
			\item allows direct coupling of signal processing with chemical actuators 
			\item application og biomolecular logic systems for analystic purposes could yield a novel class of biosensors: many input signals and binary outputs
			\item logically processed feedback between drug appl. and physiological conditions can signifacntly imprive drug targeting and efficiency 
			
			\item difficulties: complexity by assembling individual logic gates into complex logic networks (intelligent by molecular logic) (43-34-67)
			\item new approach for the sensor design and operation, interfach biocomputing system and electronic transducers
		\end{itemize}


\section{possible application}
	\begin{itemize}
		\item not just chronic but also ..........
		\item state of the art
		\item feedback loop currently devoted to management of diabetes through integration of an electrochemical glucose sensing element with an insulin-delivery feedback loop for the optimal dose of insulin (69-71)
		\item example analyse protein libraties associated with muliple sclerosis(58)
		
		ENzyme logic system recognizing various injury-related physiological conditions
		\begin{itemize}
			\item types of injuries result in concentrations of chemical substances in the body
			\\	
			\item example: lactate axidase, horeserasish peroxidase and glucose dehydrogenase = designed to process biochemical information related to pathophysiological conditions from brain injury
			\item markers: glucose(hemorrhagic shock),lactate(rhagic shock or traumic brain injury) and norepinephrine(traumatic injury)
			\item logic 0 = normal concentrations
			\item change results into different numbers 1,2,3 - convenient
			\item = biocomputing logic system 
			\item challenge: difference between normal and unnormal minimal => not linear, should be sigmoidal	
		\end{itemize}
	\end{itemize}



\section{considerations}
stabilization and confinement//
optical transduction 
\subsection{surface immobilization of the biocomputing machinery}
	
	
	
	\begin{itemize}
		\item optimal surface confinement of the biocomputing layer
		\item engineering enzyme microenvironment (transducer layer)
		\item contact between biocomputing layer and transducing surface
		\item combine individual logic-gates and maintain high enzymatic stability and reataining individual reagents
		\item leakage of cosubstrate 
		\item no cross-reactions
		\item surface confinement? layer-by layer? more efficient and rational 
		\item level of the surface confined reagents tailored for account of different input concentrations /enzyme activities 
		\item coating: optimized for transport and excluding potential interfernece and protecting the surface
	\end{itemize}

\subsection{optimal transduction of biocomputing signal processes}
	\begin{itemize}
		\item simultaneous measurements of multiple outputs require different transduction strategies (common: fixed potential )
	\end{itemize}
	\begin{itemize}
		\item Requires:interface of biocomputing systems + electronic transducer\\
		Therefore
		\item scalability (increasing nuber of logic gates, assembling into complex networks)
		\item complexity(coupling of gates abd non boolean elements)
		\item composition, preparation and immobilization of the biocomputing surface layer
		\item layer by layer
		\item optimal surface confinement 
		\item careful engineering of the enzyme microenvironment(on transducer surface) for performance
		\item biocomputing layer + transducing layer + combine individual logic-gate elements	
	\end{itemize}



\section{Conclusion}
	good but needs lot of work\\
	sums up bla


\begin{thebibliography}{8}
	\bibitem{application}
	Parikha Mehrotra, Biosensors and their applications - A Review (2016)
	
	\bibitem{biosensors}
	Daniel R. Thevenot, Klara Toth, Richard A. Durst, George S. Wilson, Electrochemical biosensors: recommended definitions and classifications In: Biosensors and Bioelectronics 16 (2001) 121- 131
	
	\bibitem{source1}
	Joseph Wang, Evgeny Katz, Digital biosensors with build-in logic for biomedical applications- biosensors based on a biocomputing concept, in: Anal Bioanal Chem(2010) 1591-1603
	
	\bibitem{source2}
	Shengbo Sang, Wendong Zhang and Yuan Zhao, State of the Art in Biosensors, in: State of the Art of Biosensors(2013) 89-110
	
	\bibitem{source3}
	Evgeny Katz, Enzyme-Based Logic Gates and Networks with Outout Signals Analyzed by Various Methods, In: ChemPhysChem 2017, 18 1688-1713
	
\end{thebibliography}
\end{document}

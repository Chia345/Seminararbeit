\documentclass[runningheads]{llncs}
\usepackage{graphicx}
\begin{document}
\title{Biosensors with build-in logic for medical applications}
\author{Maren Krafft}
\authorrunning{F. Author et al.}
\institute{Universität Passau, Passau, Germany}
\maketitle        

      
\begin{abstract}
	
% describes central points of your report
The abstract should briefly summarize the contents of the paper in
150--250 words.



\keywords{Biomolecular computing \and Biosensors  \and enzyme logic gates \and biosensing \and biocomputing systems \and enzyme logic circuits \and biomedical applications}
\end{abstract}

\section{Introduction}
%1.Context: what problem is being solved, why relevant or interesting, benefit, possible applications
\subsection{Use of biosensors}
\subsection{innovation}
\subsection{how das technique work}

\section{concept and history}
\subsection{motivation}
\subsection{concept}
	\begin{itemize}
		\item unconventional computing
		\item 
		\item
	\end{itemize}


\section{current state and future}
\subsection{Challenges}
\subsection{Possibilities}

\section{Conclusion}



\begin{thebibliography}{8}
	\bibitem{application}
	Parikha Mehrotra, Biosensors and their applications - A Review (2016)
	
	\bibitem{biosensors}
	Daniel R. Thevenot, Klara Toth, Richard A. Durst, George S. Wilson, Electrochemical biosensors: recommended definitions and classifications In: Biosensors and Bioelectronics 16 (2001) 121- 131
	
	\bibitem{source1}
	Joseph Wang, Evgeny Katz, Digital biosensors with build-in logic for biomedical applications- biosensors based on a biocomputing concept, in: Anal Bioanal Chem(2010) 1591-1603
	
	\bibitem{source2}
	Shengbo Sang, Wendong Zhang and Yuan Zhao, State of the Art in Biosensors, in: State of the Art of Biosensors(2013) 89-110
	
	\bibitem{source3}
	Evgeny Katz, Enzyme-Based Logic Gates and Networks with Outout Signals Analyzed by Various Methods, In: ChemPhysChem 2017, 18 1688-1713
	
\end{thebibliography}
\end{document}

\documentclass[runningheads]{llncs}
\usepackage{graphicx}
\begin{document}
\title{Multisignal digital biosensors- digital Biosensors integrated with enzyme logic systems}
\author{Maren Krafft}
\authorrunning{F. Author et al.}
\institute{Universität Passau, Passau, Germany}
\maketitle        

      
\begin{abstract}
	
% describes central points of your report
The abstract should briefly summarize the contents of the paper in
150--250 words.



\keywords{Biomolecular computing \and Biosensors  \and enzyme logic gates \and biosensing \and biocomputing systems \and enzyme logic circuits \and biomedical applications}
\end{abstract}

\section{Introduction}
%1.Context: what problem is being solved, why relevant or interesting, benefit, possible applications
%2.innovation: described technique completely new or does it improve earlier approcahes? what is improved 
%3.thesolution: how does the presented technique work, core idea of the solution 
	\begin{itemize}
		\item common biosensing devises are based on a single input 
		\item high-fidelity compared
		\item closed loop/feedback loops possible (sense/act/treat)
		\item rapid and reliable assessment of overall physiological condition
		\item could initiate optimal timely therapeutic intervention
		\item application og biomolecular logic systems for analystic purposes could yield a novel class of biosensors: many input signals and binary outputs
		\item logically processed feedback between drug appl. and physiological conditions can signifacntly imprive drug targeting and efficiency 
	\end{itemize}

	\begin{itemize}
		\item Biosensors + enzymes
	\end{itemize}

\section{concept}

\begin{itemize}
	\item Biosensors: layers
	\item Biocomputing: concept
	\item combination concept
		\section{Biosensors with Biocomputing = biochemical logic systems}
		Biosensors logically processing multiple biochemical signals\\
		-such procassed information produces a final output yes/no \\
		- boolean logic networks composed of biomolecular systems\\
		
		\begin{itemize}
			\item multiple target analytes(inputs) for enzyme gates
			\item high-fidelity compared
			\item closed loop/feedback loops possible (sense/act/treat)
			\item rapid and reliable assessment of overall physiological condition
			\item could initiate optimal timely therapeutic intervention
			\item biosensors + enzyme logic gates
			\item allows direct coupling of signal processing with chemical actuators 
			\item application og biomolecular logic systems for analystic purposes could yield a novel class of biosensors: many input signals and binary outputs
			\item logically processed feedback between drug appl. and physiological conditions can signifacntly imprive drug targeting and efficiency 
			
			\item difficulties: complexity by assembling individual logic gates into complex logic networks (intelligent by molecular logic) (43-34-67)
			\item new approach for the sensor design and operation, interfach biocomputing system and electronic transducers
		\end{itemize}
	\item example
\end{itemize}

\section{possible application}
	\begin{itemize}
		\item state of the art
		\item diabetes
		\item scelerosis 
		\item brain trauma
	\end{itemize}



\section{considerations}
	\begin{itemize}
		\item Requires:interface of biocomputing systems + electronic transducer\\
		Therefore
		\item scalability (increasing nuber of logic gates)
		\item complexity(coupling of gates abd non boolean elements)
		\item composition, preparation and immobilization of the biocomputing surface layer
		\item layer by layer
		\item optimal surface confinement 
		\item careful engineering of the enzyme microenvironment(on transducer surface) for performance
		\item biocomputing layer + transducing layer + combine individual logic-gate elements	
	\end{itemize}



\section{Conclusion}
	good but needslot of work
	sums up bla


 
\section{Biosensors}
	\begin{itemize}
		\item single input
	\end{itemize}
\section{Biosensors with Biocomputing = biochemical logic systems}
	Biosensors logically processing multiple biochemical signals\\
	-such procassed information produces a final output yes/no \\
	- boolean logic networks composed of biomolecular systems\\
	
	\begin{itemize}
		\item multiple target analytes(inputs) for enzyme gates
		\item high-fidelity compared
		\item closed loop/feedback loops possible (sense/act/treat)
		\item rapid and reliable assessment of overall physiological condition
		\item could initiate optimal timely therapeutic intervention
		\item biosensors + enzyme logic gates
		\item allows direct coupling of signal processing with chemical actuators 
		\item application og biomolecular logic systems for analystic purposes could yield a novel class of biosensors: many input signals and binary outputs
		\item logically processed feedback between drug appl. and physiological conditions can signifacntly imprive drug targeting and efficiency 
		
		\item difficulties: complexity by assembling individual logic gates into complex logic networks (intelligent by molecular logic) (43-34-67)
		\item new approach for the sensor design and operation, interfach biocomputing system and electronic transducers
	\end{itemize}

Beispiel
	\begin{itemize}
		\item Analyze protein libraries associated with muliple sclerosis 
	\end{itemize}
considerations
	
\section{biocomputing}
	\begin{itemize}
		\item most represent only the proof of the concept /possibility of performing computing wit use of biomolecular systems 
		\item not practial
		\item biomolecular logic systems
		\item challenges sclaing ip the complexity of the systems by networking the indicidual parts of a logic circuit 
		
	\end{itemize}

\begin{thebibliography}{8}
	\bibitem{application}
	Parikha Mehrotra, Biosensors and their applications - A Review (2016)
	
	\bibitem{biosensors}
	Daniel R. Thevenot, Klara Toth, Richard A. Durst, George S. Wilson, Electrochemical biosensors: recommended definitions and classifications In: Biosensors and Bioelectronics 16 (2001) 121- 131
	
	\bibitem{source1}
	Joseph Wang, Evgeny Katz, Digital biosensors with build-in logic for biomedical applications- biosensors based on a biocomputing concept, in: Anal Bioanal Chem(2010) 1591-1603
	
	\bibitem{source2}
	Shengbo Sang, Wendong Zhang and Yuan Zhao, State of the Art in Biosensors, in: State of the Art of Biosensors(2013) 89-110
	
	\bibitem{source3}
	Evgeny Katz, Enzyme-Based Logic Gates and Networks with Outout Signals Analyzed by Various Methods, In: ChemPhysChem 2017, 18 1688-1713
	
\end{thebibliography}
\end{document}

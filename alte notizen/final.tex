\documentclass[runningheads]{llncs}
\usepackage{graphicx}
\begin{document}
\title{Biosensors with build-in logic for medical applications}
\author{Maren Krafft}
\authorrunning{F. Author et al.}
\institute{Universität Passau, Passau, Germany}
\maketitle        

      
\begin{abstract}
The abstract should briefly summarize the contents of the paper in
150--250 words.

\item general concept 
\item multisignal digital biosensors processing complex pattern of different physiological markers
\item practival considerations
\item challenges


\keywords{First keyword  \and Second keyword \and Another keyword.}
\end{abstract}

\section{Introduction}

	overviw of the novel research paradigm of digitally operating biosensors logically processing multiple biochemical signals through Boolean logic networks composed of biomolecular systems 
\subsection{Use of biosensors}
\subsection{innovation}
\subsection{how das technique work}



%Chancen
\section{Chancen}
\begin{itemize}
	\item biomedical monitoring(example, closed-loop-> patient tailored timely therapy, personalized medicine= sensing devices + delivery devices)
	\begin{itemize}
		\item closed-loop -> patient tailored timely therapy possible
		\item sensing devices + delivery devices = personalized medicine
		\item example feedback-loop: diabetes: electrochemical glucose sensing element + insulin-delivery feedback loop
		\item fast delivery in emergencies 
	\end{itemize}
	\item environmental monitoring
	\item national defense
	\item food safety
	\\
	\end{itemize}





\section{concept}
	- Defintion Biosensor:
	- Build-in logic:
	- Def unconventional computing and contrast to conventional
	- kurz was sind die main Boolean operations used: AND, OR...
	- biocomputing 
	- multiple biochemical coupling of signal processing with chemical actuators

\subsection{Biocomputing}
	Definition
	\begin{itemize}
		\item single logic gates (mimicking Boolean) to small logic networks
		\item biomolecular systems for processing chemical information
		\item more complex than nonbiological systems 
		\item coupling enzymatic reactions (logic gates) with electronic transducers and signal-responing materials 
	\end{itemize}

	=>enzyme logic gates

\subsection{Enzyme logic gates}
	\begin{itemize}
		\item enzymatic reactions
		\item coupling of logic gates with electronic transducers and signal responsive materials
		\item transducers:
		\item signal responive material
		
	\end{itemize}
\subsubsection{theoretical}
	
	\begin{itemize}
		\item glucose oxidase and catalyse operating as logic gates: 
		\item input : H2O2 and glucose
		\item gluconic acid = biocalatyltic oxidation of glucose 
		\item only when both present opical output signal. = AND
		\item define logic values: small changes = 0 and large absorbance changes as 1 => AND
		\item similar possible with XOR,AND, OR, NOR, INHIBIT
		\item with logic gates with modular structur that enables therir assembly in networks NAND/ NOR possible
		\item logic gates and their networks = biomolecular information processing systems 
		\item => biosensoric systems with logically processed signals represented by various biomarkers(characteristic of different abnormal physiological conditions)
	\end{itemize}

\subsubsection{example ph}
	\begin{itemize}
		\item pH changes in solution as logic respones to input signals
		\item AND invertase + glucose oxidase (from 5.8 to 3.5)
		\item OR ersterase and glucose oxidase in glucose and ethyl butyrate - when one of both present ->acidification  
		\item neutral ph = 3,5
	\end{itemize}
	Conclusion: 
	\begin{itemize}
		\item don't solve real computing problem  nor operate as useful biosensors 
		\item represent first step toward the development of digital biosensors 
		\item funfact optimization of enzymatic reaction, up to 10 logic gates concatenated with low noise in the system 
	\end{itemize}


\subsubsection{ENzyme logic system recognizing various injury-related physiological conditions}
	\begin{itemize}
		\item types of injuries result in concentrations of chemical substances in the body
		\\	
		\item example: lactate axidase, horeserasish peroxidase and glucose dehydrogenase = designed to process biochemical information related to pathophysiological conditions from brain injury
		\item markers: glucose(hemorrhagic shock),lactate(rhagic shock or traumic brain injury) and norepinephrine(traumatic injury)
		\item logic 0 = normal concentrations
		\item change results into different numbers 1,2,3 - convenient
		\item = biocomputing logic system 
		\item challenge: difference between normal and unnormal minimal => not linear, should be sigmoidal	
	\end{itemize}





\section{challenges}
	prospects
	fundamental and practical challenges
	
	\begin{itemize}
		\item attention to composition, preparation and immobilization of the biocomputing surface layer
		\item sucess depends in part on immobilization of the biocomputing reagent layer
		\item system scalability 
		\item efficient transduction of the output signals
		\item high fidelity
	\end{itemize}
	stuff missing
	
	
	da war was mit wie viel können hintereinander geschaltet werden
	


\subsection{Enzyme logic circuits-scaling up the system complexity}
	\begin{itemize}
		\item main challend: scaling up the complexity of the systems by networking the individual parts of a logic circuit 
		\item addressed experimentallly when designing networks composed of concatenated enzyme logic gates
		\item assembled logic networks analyzed theoretically for opimization and noise reduction; coupling output signals with electronic transducers and bioelectronic devices
	\end{itemize}


\subsection{Biomolecular logic gates designed for biomedical analytical applications}
	problems not addressed yet in biomolecular information processing systems:
	\begin{itemize}
		\item logic 0 values were defined as the absense of the enzymes
		\item logic 1 not always correspond to the concentration expected in vivo /not normal physiological concentrations 
		\item input not justifies to their biomedical meaning
		
	\end{itemize}





\section{advantages}

	\begin{itemize}	
		\item multiple target analytes(inputs) (enzymgates)/biochemical inputs 
		\item high-fidelity biosensing compared 
		\item rapid and reliable assessment of physiological condition (enzymes + automatically provessing)
		\item optimal timely therapeutic intervention
		\item realization of closed-loop systems (sense/act/treat)
		\item missing example fields like diabetes (siehe text)	
	\end{itemize}




\section{Conclusion} 




\section{Idee des Aufbaus}
\begin{itemize}
	\item Abstract:
	Biosensoren, logic gates, possibilities, challenges and future work
	\item introduction
	Definition, possible fields(examples), special,... loops
	\item The idea 
	Theory and practical
	\item Two examples 
	\item Layers, challenges and future work
	\item conclusion
\end{itemize}


\begin{thebibliography}{8}
	\bibitem{application}
	Parikha Mehrotra, Biosensors and their applications - A Review (2016)
	
	\bibitem{biosensors}
	Daniel R. Thevenot, Klara Toth, Richard A. Durst, George S. Wilson, Electrochemical biosensors: recommended definitions and classifications In: Biosensors and Bioelectronics 16 (2001) 121- 131
	
	\bibitem{source1}
	Joseph Wang, Evgeny Katz, Digital biosensors with build-in logic for biomedical applications- biosensors based on a biocomputing concept, in: Anal Bioanal Chem(2010) 1591-1603
	
\end{thebibliography}
\end{document}
